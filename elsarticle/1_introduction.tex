% !TEX root = main.tex

Buildings are at the heart of society and currently account for 32\% of global final energy consumption and 19\% of energy related greenhouse gas emissions \cite{IPCC}. Nevertheless the building sector has a 50-90\% emission reduction potential using existing technologies, and widespread implementation could see energy use in buildings stabalise or even fall by 2050. Within this strategy, building integrated photovoltaics (BIPV) has the potential of providing a substantial segment of a buildings energy needs \cite{Shoen1997}. Even the photovoltaic (PV) industry has identified BIPV as one of the four key factors for the future success of PV \cite{raugei2009life}. \\

The PV industry is currently dominated by crystaline silicon photovoltaic cells due to their high efficiency and low processing costs \cite{saga2010advances}. However these technologies are often demoted as pre frabricated eco-spoilers that ruin the architectural integrity of a building \cite{Lueling2009ee}. This combined with their intrinsic weight restricts their large scale implementation to roofs where they are out of sight. In the last decade however, we have seen an interesting development of second generation thin film technologies \cite{NREL}. In particular, Cu(In,Ga)Se$_2$ (CIGS) which is reaching  competitive levels of efficiencies \cite{kushiya2014cis}, and manufacturing costs \cite{kaelin2004low} \cite{jelle2012building}.\\

This development has brought new BIPV design possibilities. Their light weight nature and customisable shapes allows for easier and more aesthetically pleasing integration into the building envelope. Furthermore, this technology can be easily actuated and used as an adaptive building envelope \cite{rossi2012adaptive}. \\

Adaptive buildings envelopes have gained interest in recent years due to their energy saving potentials \cite{loonen2013climate}. From an energetic perspective, the envelope acts as a buffer between the interior and exterior environments. An adaptive building envelope can mediate solar isolation on the building, thereby offering reductions in heating/cooling loads and improvement of daylight distribution \cite{rossi2012adaptive}. Interestingly the mechanics that actuate adaptive envelopes couples seamlessly with the mechanics required for facade integrated PV solar tracking. The balance of electricity production, and adaptive shading can in some cases offset the entire energy demand of an office space behind this adaptive envelope \cite{jayathissa2015abs}. We call this combination of technologies an Adaptive Solar Facade (ASF).\\

[Add Figure from nagy et al]\\

The design of an ASF comes at an added cost. The additional electronics, actuators, and supporting structure adds further embodied CO2 to the Balance of Systems (BOS). In this paper we will discuss the ASF from a life cycle perspective thus analysing whether the increase in operational savings offsets the increased embodied carbon. \\
The remainder of the paper is organised as follows. In the next section we describe the inventory of an ASF, the operational emissions and LCA methadology used in this analysis. In Section \ref{ch:results} we will present the results from the LCA and look at the major sources of embodied carbon, along with the operational performance. In Section \ref{ch:comparison} we compare the results of the LCA with other technologies and the effects in different regions. In Section \ref{ch:discussion} we discuss the results and possible implications this will have on the future design of Adaptive Solar Facades. 



%where existing BIPV products are catergorised as foils, tiles modules and glazing products \cite{jelle2012building}. Although the effeciencies of these systems 






% - In the last decades, building integrated photovoltaics (BIPV) have been adopted as part of the energy strategy towards 2050... \

% (advantages of BIPV, potential of BIPV)\\


% - The current developments of light-weight efficient thin film technologies have brought new design possibilities for architects in BIPV design... \

% (Adaptive Building Envelopes, examples, Envelope is the barrier between the internal and external environment, Advantages, seamless coupling with solar tracking mechanics) \\

% - One example of a multi functional facade that was recently released is the Adaptive Solar Facade.\\

% - The aim of this paper is to analyse the life cycle emissions of an adaptive solar facade and provide comparisons with standard shading systems and static BIPV solutions.\\
