% !TEX root = main.tex

The adaptive shading qualities of the ASF have added savings to the heating, cooling and lighting electricity consumption of the building. These savings are three times greater than the initial embodied carbon. When we subtract these savings from the embodied energy, we have a negative emission factor of \textcolor{red}{-470.1 gCO2/kWH}. This demonstrates the advantage of using the PV material, not only as an electricity generation unit, but also as a building material for adaptive shading systems. On the conrast, the ASF without any shading benefits has an emission factor approximately \textcolor{red}{6 times } greater than a standard CIGS solution. This is primarily due to the added electronics, actuators, and a larger supporting structure. Furthermore the photovoltaic electricity generation of the PV panels is lower due to aspects of self shading \cite{hofer2015photovoltaics}. 

The operational carbon savings through shading however is sensitive to the country where it is in operation. A country with a low GWP electricity mix will result in lower operational carbon savings than a country with a high GWP electricity mix. For example, an ASF installed in Switzerland only has \textcolor{red}{XXXkgCO2} over its lifetime through shading which results in an end emission factor of \textcolor{red}{174 g CO2/kWh}. Germany on the otherhand would have an emission factor of \textcolor{red}{-700 g CO2/kWh}.\\

Although it is favorable to install an ASF in Germany, it has still benefits in countries such as Switzerland. For instance, with an emission factor 6\% less than the standard mix, it contributes to a nuclear free energy mix. Furthermore it provides interesting design options for architects where they can install PV in locations which were previously not possible, which helps increase the BIPV potential.  \\

When designing an ASF architects and engineers may consider: 
\begin{itemize}
\item The use of soft robotic actuators over servo motors saves \textcolor{red}{60kgCO2/sqm}
\item Control system electronics cost \textcolor{red}{XXkgCO2/kg} and therefore should be carefully designed
\item The structural support system in our current analysis used a stainless steel frame representing \textcolor{red}{20\%} of our total embodied carbon emissions. Using plain steel, or an alternative material with a lower GWP should be considered
\item If the ASF is installed over a opaque building surface then the advantages of adaptive shading are not present. In this case, a static system is a preferred design choice. 
\end{itemize}


One limitation of the LCA is that the locations are bound to the climate of where the energy simulation was conducted. In our case, the simulation was conducted in Geneva, Switzerland. This restricts our LCA to similar temperate climates on the same latitude. Conducting the simulation in Spain, the MENA region, or a tropical climate may lead to different results and conclusions.

% The major limitation of the current ASF designs is the large BOS as seen in Figure \ref{fig:embodied}. There is a large room for improvement for architects and engineers to better optimise the system. The choice of actuation system, for example, has a significant on the embodied emissions. The GWP of a Soft Robotic Actuator including the air compressor and tubing comes to \textcolor{red}{20kgCO2/sqm} per unit whereas a classical servo motor is four times greater at \textcolor{red}{80kgCO2/sqm}. Also, aspects such as the supporting structure can be better designed to use less steel, or an alternative material with a lower GWP. The design of the control system should also be carefully thought out due to the high GWP of electronic components. The control system required for an ASF where each panel can be independently actuated emits \textcolor{red}{35 \%} more CO2 as a control system where only rows are independently actuated. 





% \textcolor{green}{ [Emission factor without shading: 307.6gCO2/kWh
% - When is the ASF advantages, when is it not
% - Would it be better to just have a optimally angled static system?
% - Limitations of the study
% - Nuclear power in France and Switzerland
% - No need to purchase land, advantage of facade integration
% - What should designers of adaptive solar facades keep in mind 
% - Other advantages of the ASF that are not clear in the LCA analysis, such as daylighting and user centered control 
% - Have a technology where you can put PV where you normally can't put PV
% - Choice of actuator]}