% !TEX root = main.tex



As seen in Figure \ref{fig:compPV}, an ASF without any shading benefits has an emission factor approximately \textcolor{red}{6 times } greater than a standard CIGS solution. This is primarily due to the added electronics, actuators, and a larger supporting structure as seen in Figure \ref{fig:embodied}. Furthermore the photovoltaic electricity generation of the PV panels is lower due to aspects of self shading [REF Johannes]. However, when we look at the facade with adaptive shading benefits then we actually have a BIPV solution that has a negative emission factor. The savings to the buildings energy consumption through heating cooling and lighting offsets the embodied GWP threefold. \\

This value however, is very sensitive to the electricity mix where the analysis is being conducted. If the ASF was installed in an area with a low GWP electricity mix such as Switzerland then the emission factor increases to \textcolor{red}{174 g CO2/kWh}. \\

[maybe add a waterfall chart of Switzerland]\\

Although it is worse to have an ASF installation in Switzerland compared to Germany, it does still have positive effects that aren't highlighted in the study. The ASF still has an emission factor 6\% lower than the electricity mix, and provides means of using less nuclear power. Furthermore it provides interesting design options for architects where they can install PV in locations which were previously not possible.  \\

The major limitation of the current ASF designs is the large BOS as seen in Figure \ref{fig:embodied}. Unlike the PV panels, there is a large room for improvement for architects and engineers to better optimise the system. The choice of actuation system for example has a significant / minimal impact on the embodied emissions. [Elaborate further when results come in]. Also, aspects such as the supporting structure can be better designed to use less steel.





Emission factor without shading: 307.6gCO2/kWh

- When is the ASF advantages, when is it not

- Would it be better to just have a optimally angled static system?

- Limitations of the study

- Nuclear power in France and Switzerland

- No need to purchase land, advantage of facade integration

- What should designers of adaptive solar facades keep in mind 

- Other advantages of the ASF that are not clear in the LCA analysis, such as daylighting and user centered control 

- Have a technology where you can put PV where you normally can't put PV

- Choice of actuator