% !TEX root = main.tex

An adaptive solar facade, purely as a solar tracking and electricity generation technology is inferior to simpler static solutions. Its emission factor is approximately twice and three times worse than Poly-Si and CIS solutions respectively. This is due to the additional GWP of the control system, supporting structure, actuators, and the energy required for actuation. \\

However when we also consider the energy savings to the building through adaptive shading we have a system that yields a negative emission factor of -465.3 gCO${_2}$/kWh. This is because the savings to the building system in terms of heating, cooling, and lighting offsets the embodied GWP threefold. This demonstrates the advantage of using the PV material, not only as an electricity generation unit, but also as a building material for adaptive shading systems.\\




%The adaptive shading qualities of the ASF reduce the heating, cooling and lighting electricity consumption of the building, which translates into reductions in GWP based off the electricity mix. These savings are three times greater than the GWP emissions implied by the materials. Taking these savings into account yields a negative emission factor of -465.3 gCO${_2}$/kWh. This demonstrates the advantage of using the PV material, not only as an electricity generation unit, but also as a building material for adaptive shading systems. On the contrary, the ASF without any shading benefits has an emission factor approximately \textcolor{red}{6 times} greater than a standard CIGS solution. This is primarily due to the added electronics, actuators, and a larger supporting structure. Furthermore the photovoltaic electricity generation of an ASF is lower due to aspects of self shading \cite{hofer2015photovoltaics}. 

The GWP savings through adaptive shading however is sensitive to the GWP of the electricity mix. A country with a low GWP electricity mix will result in lower operational GWP savings than a country with a high GWP electricity mix. For example, an ASF installed in Switzerland only has an emission factor of \textcolor{red}{80.7 g CO2/kWh}. Germany on the otherhand would have an emission factor of \textcolor{red}{-700 g CO2/kWh}.\\

Although it is favorable to install an ASF in Germany, it has still benefits in countries such as Switzerland. For instance, with an emission factor 6\% less than the standard mix, it contributes to a nuclear free energy mix. Furthermore it provides interesting design options for architects where they can install PV in locations which were previously not possible. This increases BIPV potential.  \\

When designing an ASF architects and engineers may consider: 
\begin{itemize}
\item The trade off between soft robotic actuators and servo motors for actuation. Although a soft robotic actuator has an embodied GWP three times lower than a servo motor, it requires 150 times more energy to actuate. Purely from an LCA perspective, if more than three actuations are required a day, servo motors would be the preferred solution. 
\item Control system electronics cost \textcolor{red}{XXkgCO2/kg} and therefore should be carefully designed.
\item The structural support system in our current analysis used a stainless steel frame representing 20.4\% of our total embodied carbon emissions. Using plain steel, or an alternative material with a lower GWP should be considered.
\item If the ASF is installed over a opaque building surface then the advantages of adaptive shading are not present. In this case, a static system is a preferred design choice. 
\end{itemize}


One limitation of the LCA is that the locations are bound to the climate of where the energy simulation was conducted. In our case, the simulation was conducted in Geneva, Switzerland. This restricts our LCA to similar temperate climates on the same latitude. Conducting the simulation in Spain, the MENA region, or a tropical climate may lead to different results and conclusions. Furthermore the analysis focusses on a single office room. Expanding the analysis to the entire building, or urban level may yield different results.\\
The LCA also assumes that the ENTSO-E mix will remain constant over the next 20 years. With the growth of the renewable energy sector, we will observe an increase in the emission factor of ASF systems. \\
The LCA also excludes other aspects of the ASF system such as the reduction of heating and cooling systems themselves, and the increase in user comfort. 

\textcolor{magenta}{
\\Further thoughts
\begin{itemize}
\item only one room investigated
\item Other impacts (e.g. Recipe) to be investigated and lead to very different conclusions
\end{itemize}
}

% The major limitation of the current ASF designs is the large BOS as seen in Figure \ref{fig:embodied}. There is a large room for improvement for architects and engineers to better optimise the system. The choice of actuation system, for example, has a significant on the embodied emissions. The GWP of a Soft Robotic Actuator including the air compressor and tubing comes to \textcolor{red}{20kgCO2/sqm} per unit whereas a classical servo motor is four times greater at \textcolor{red}{80kgCO2/sqm}. Also, aspects such as the supporting structure can be better designed to use less steel, or an alternative material with a lower GWP. The design of the control system should also be carefully thought out due to the high GWP of electronic components. The control system required for an ASF where each panel can be independently actuated emits \textcolor{red}{35 \%} more CO2 as a control system where only rows are independently actuated. 

% \textcolor{green}{ [Emission factor without shading: 307.6gCO2/kWh
% - When is the ASF advantages, when is it not
% - Would it be better to just have a optimally angled static system?
% - Limitations of the study
% - Nuclear power in France and Switzerland
% - No need to purchase land, advantage of facade integration
% - What should designers of adaptive solar facades keep in mind 
% - Other advantages of the ASF that are not clear in the LCA analysis, such as daylighting and user centered control 
% - Have a technology where you can put PV where you normally can't put PV
% - Choice of actuator]}