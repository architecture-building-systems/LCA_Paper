% !TEX root = main.tex



As seen in Figure \ref{fig:compPV}, an ASF without any shading benefits has an emission factor approximately \textcolor{red}{6 times } greater than a standard CIGS solution. This is primarily due to the added electronics, actuators, and a larger supporting structure as seen in Figure \ref{fig:embodied}. Furthermore the photovoltaic electricity generation of the PV panels is lower due to aspects of self shading [REF Johannes]. However the adaptive shading qualities of the ASF has added savings to the buildings heating, cooling and lighting electricity consumption. These savings offsets the initial embodied carbon three fold. When we subtract these savings from the embodied energy as seen in Figure \ref{fig:waterfall}, we have a negative emission factor of \textcolor{red}{-470.1 gCO2/kWH}. 

The operational carbon savings through shading however is sensitive to the country where it is in operation. A country with a high GWP electricity mix will result in higher operational carbon savings than a country with a low GWP electricity mix. For example, an ASF installed in Switzerland only has XXXkgCO2 over its lifetime through shading which results in an end emission factor of \textcolor{red}{174 g CO2/kWh}. \\



Although it is worse to have an ASF installation in Switzerland compared to Germany, it does still have positive effects that aren't highlighted in the study. The ASF still has an emission factor 6\% lower than the electricity mix, and provides means of using less nuclear power. Furthermore it provides interesting design options for architects where they can install PV in locations which were previously not possible.  \\

The major limitation of the current ASF designs is the large BOS as seen in Figure \ref{fig:embodied}. There is a large room for improvement for architects and engineers to better optimise the system. The choice of actuation system, for example, has a significant on the embodied emissions. The GWP of a Soft Robotic Actuator including the air compressor and tubing comes to \textcolor{red}{20kgCO2/sqm} per unit whereas a classical servo motor is four times greater at \textcolor{red}{80kgCO2/sqm}. Also, aspects such as the supporting structure can be better designed to use less steel, or an alternative material with a lower GWP. The design of the control system should also be carefully thought out due to the high GWP of electronic components. The control system required for an ASF where each panel can be independently actuated emits \textcolor{red}{35 \%} more CO2 as a control system where only rows are independently actuated. 





[[Emission factor without shading: 307.6gCO2/kWh

- When is the ASF advantages, when is it not

- Would it be better to just have a optimally angled static system?

- Limitations of the study

- Nuclear power in France and Switzerland

- No need to purchase land, advantage of facade integration

- What should designers of adaptive solar facades keep in mind 

- Other advantages of the ASF that are not clear in the LCA analysis, such as daylighting and user centered control 

- Have a technology where you can put PV where you normally can't put PV

- Choice of actuator]]