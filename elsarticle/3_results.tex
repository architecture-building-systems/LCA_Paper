% !TEX root = main.tex

We present the results of the LCA analysis in relation to the 1) embodied emissions, 2) operational savings through our energy plus simulation, 3) a calculation of the emission factor, 4) global distribution of GWP, 5) regional effects of terrestrial acidification, 6) sensitivity of the LCA to design and location, and 6) a comparison to other PV technologies.

\subsection{LCA of the Adaptive Solar Facade}

A breakdown of the embodied carbon emissions can be found in Figure  \ref{fig:embodied}. The largest embodied global warming potential (GWP) contribution in the ASF comes from the solar panels, the electronics and the supporting structure. \textcolor{red}{Change steel frame to supporting structure in figures}. 

\begin{figure}[H]
\begin{center}
\includegraphics[width=10cm, trim= 0cm 0cm 0cm 0cm,clip]{pieembodied}
\caption{Breakdown of the embodied carbon emissions, it can be seen that xxxx has the greatest GWP contribution}
\label{fig:embodied}
\end{center}
\end{figure}

\subsection{Operational Savings through Adaptive Shading}
\label{ch:oppResults}

The operational energy savings through adaptive shading was simulated in Energy Plus as explained in Section \ref{ch:method}. We calculated a total energy saving of 25\% compared to louvers at 45$^\circ$ and 56\% compared to a case with no facade shading \cite{jayathissa2015abs}. These results are sumarised in Figure \ref{fig:operational}.


\begin{figure}[H]
\begin{center}
\includegraphics[width=10cm, trim= 0cm 0cm 0cm 0cm,clip]{buildingenergy.pdf}
\caption{Breakdown of the opperational carbon emissions for system with a) no shading, b) with louvers at 45$^\circ$ and c) with an ASF -- not including onsite electricity production.}
\label{fig:operational}
\end{center}
\end{figure}

The total GWP of the ASF based off Equation \ref{eq:GWP}, can be built up using a waterfall chart, Figure \ref{fig:waterfall}. The total emission factor of electricity produced by the ASF comes to (-470.1 g CO$_2$-eq./kWh).\\

\begin{figure}[H]
\begin{center}
\includegraphics[width=14cm, trim= 0cm 0cm 0cm 0cm,clip]{waterfall}
\caption{Waterfall diagram of GWP of the ASF. The far left column details the embodied carbon emissions. The second bar details the emission reduction of the building through the smart shading algorithms of the ASF. The third column shows an increase of emissions through maintenance. The fourth column shows an increase in emissions in the disposal. This leaves us with a final emissions value. When we apply this value to Equation \ref{eq:solar}, we obtain an emission factor per kWh of -472.8 g CO$_2$-eq./kWh.}
\label{fig:waterfall}
\end{center}
\end{figure}

\subsection{Global Distribution of GWP and Terrestrial Acidification}

The global distribution of embodied GWP emissions is focused in Europe, specifically Germany and Switzerland as most of the manufacturing is done in this region. It can be see however that emissions occur globally due the sourcing of primary materials from many locations around the world. Terrestrial acidification however is more interesting as it has a local impact compared to carbon emissions. It is interesting to note that China carries the greatest burden of terrestrial acidification from the ASF production. 
\begin{figure}[H]
\begin{center}
\includegraphics[width=10cm, trim= 0cm 0cm 0cm 0cm,clip]{mapGWP.pdf}
\caption{Global distribution of embodied GWP emissions}
\label{fig:mapGWP}
\end{center}
\end{figure}

\begin{figure}[H]
\begin{center}
\includegraphics[width=10cm, trim= 0cm 0cm 0cm 0cm,clip]{mapacidification.pdf}
\caption{Global distribution of terrestrial acidification}
\label{fig:mapAcid}
\end{center}
\end{figure}



\subsection{Sensitivity Analysis}

The results of the sensitivity analysis is shown in Figure \ref{fig:sens}. 

(Might Bring this to the discussion)\\
The operational GWP savings is dependent on the electricity mix as explained in Section \ref{ch:Meth:Opp}. Changing our assumption from the European ENTSO-E mix to a country specific mix brings interesting results. In Switzerland, the mix is dominated by hydro and nuclear power which has a very low GWP potential [citation needed\textcolor{magenta}{ \textit{ecoinvent?}}]. This would then increase the emission factor to (100gCO2/kWh), which is only 6\% higher than the emission factor in Switzerland. The German mix on the other hand has a higher GWP mix than the ENTSO-E mix due to their high share in coal fire plants. This then reducing our emission factor futher [Double Check this].\\

The choice of actuator also has a large impact on the GWP of the embodied carbon. A single Soft Robotic Actuator, including the air compressor and tubing comes to \textcolor{red}{20kgCO2/sqm} per unit whereas a classical servo motor is four times greater at \textcolor{red}{80kgCO2/sqm}.\\

The control system design should be carefully through out due to the high GWP of electronic components [REFERENCE REQUIRED]. The control system required for an ASF where each panel can be independently actuated emits \textcolor{red}{35 \%} more CO2 as a control system where only rows are independently actuated.\\

\textcolor{red}{The Monte Carlo Simulation did not yield any significant results / was significant. } \textcolor{blue}{Fill this out once your results come in}


% In Switzerland, we see a 6\% reduction compared to the average electricity mix. This is because the Swiss electricity mix is dominated by hydro and nuclear power which has a very low GWP potential [citation needed\textcolor{magenta}{ \textit{ecoinvent?}}]. In Germany on the other hand, the ASF has a 81\% reduction in carbon emissions as the emission factor of the electricity grid is roughly five times higher compared to Switzerland [citation needed] due to the relatively high share in coal-fired power plants.\\


\begin{figure}[H]
\begin{center}
\includegraphics[width=10cm, trim= 0cm 0cm 0cm 0cm,clip]{sens.pdf}
\caption{Sensitivity analysis of the emission factor based on electricity mix, actuation system, control system, and Monte Carlo Analysis \textcolor{red}{Wrong Figure)}}
\label{fig:sens}
\end{center}
\end{figure}



\subsection{Comparison to existing PV technologies}

Comparison of the ASF to other PV technologies and the ENTSO-E electricity mix is highlighted in Figure \ref{fig:compPV}. Because the GWP savings through adaptive shading offsets the entire embodied GWP three fold, we have a system that has a negative emission factor. It is this multifucntional nature that makes it superior to other technological choices. However, if the benefits of adaptive shading are not present, i.e. it is mounted on an opaque building surface, then we see an under performance to other static PV technologies. 

Figure \ref{fig:compPV} also highlights a case in Switzerland where the electricity mix has a low GWP. It is capable of out performing Silicone based technologies but is still inferior to simply mounted CIGS panels. Note that the just panels of the ASF, without the BOS, still has a higher emission factor than the CIGS installation. This is due to lower power production as a result of self shading \cite{hofer2015photovoltaics}, and situations where the panels are not optimally positioned.

\begin{figure}[H]
\begin{center}
\includegraphics[width=10cm, trim= 0cm 0cm 0cm 0cm,clip]{compPV.pdf}
\caption{Comparison of thin-film and BOS to other PV technologies. I would add some extra columns, one without shading, one with the ASF in Switzerland, one with the ASF in Europe}
\label{fig:compPV}
\end{center}
\end{figure}

% \begin{figure}[H]
% \begin{center}
% \includegraphics[width=10cm, trim= 0cm 0cm 0cm 0cm,clip]{regionGridMix.pdf}
% \caption{Comparison of the ASF when in operated in Switzerland when compared to other countries in the world that have similar climate conditions. Note that the data will have to change because we can't really compare (ASF in Switzerland) with Germany. We should be comparing the ASF in Germany with Germany}
% \label{fig:compReg}
% \end{center}
% \end{figure}



% - As input parameters of production processes are stochastic, a Monte Carlo simulation is used to include this stochastic behavior in the results, as shown in Figure \ref{fig:monte}...

% \begin{figure}[H]
% \begin{center}
% \includegraphics[width=10cm, trim= 0cm 0cm 0cm 0cm,clip]{monte}
% \caption{Monte carlo simulation based on input uncertainties}
% \label{fig:monte}
% \end{center}
% \end{figure}

% - Sourcing location greatly influences the embodied GWP. For photovoltaic panels, the majority of embodied emissions result from the use of electricity during production. The GWP per kwh of the Chinese electricity mix is 1145.8 ${\mathrm{gCO_2eq/kWh}}$, while in Switzerland this is only 119.6 ${\mathrm{gCO_2eq/kWh}}$.
% % cool - did we already discuss what other sensitivities to use?

% \begin{figure}[H]
% \begin{center}
% \includegraphics[width=10cm, trim= 0cm 0cm 0cm 0cm,clip]{sensitivity}
% \caption{Sensitivity analysis based on sourcing location}
% \label{fig:sensitivity}
% \end{center}
% \end{figure}
