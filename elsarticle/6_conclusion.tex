% !TEX root = main.tex

The environmental performance of the ASF has been shown to be favorably competitive with a classic BIPV installation. Even though the embodied environmental performance is \textcolor{red}{six times} worse than a static CIGS installation, it is offset three fold through adaptive shading. This combination of adaptive shading with photovoltaic generation brings new advantages to solar as it effectively has a negative emission factor of \textcolor{red}{-470.1 g CO2/kWh}. These advantages however, will not be present if the ASF is installed over an opaque building surface. It is therefore better to install static systems over opaque facades, and keep the adaptive system for glazed facades only. \\
The design of an ASF naturally can greatly influence the results. The use of soft robotic actuators over servo motors increases the environnmental performance by xxx per square meter. The control electronics, which represents 11\% of the total embodied emissions, can increase by \textcolor{red}{35\%} if we increase the resolution of the facade so each panel can move independently. Also the supporting structure, representing \textcolor{red}{20\%} of total emissions can be better optimised to use less steel, or an alternative material. \\
The emission factor of the grid electricity mix also varies greatly between countries and has a large influence on the savings through adaptive shading. The reductions in heating, cooling and lighting of an office room will have CO2 savings based on the electricity mix. In a country with low CO2 grid mix, such as Switzerland, will only see YY reductions. In Germany on the other hand, there will be savings of YY.
Ultimately we see that the combination of PV production and adaptive shading, compliment each other nicely. We see an improvement in environmental performance of PV technology, and open up new architectural possibilities for the aesthetic integration of PV panels over glazed surfaces.


