% !TEX root = main.tex

The environmental performance of the ASF has been shown to be favorably competitive with a classic BIPV installation. Even though the total embodied for the ASF are high, this is offset through energy savings by adaptive shading. This combination of adaptive shading with photovoltaic generation brings new advantages to solar as it effectively has a negative emission factor of -537.0 gCO${_2}$/kWh. These advantages however, will not be present if the ASF is installed over an opaque building surface. It is therefore preferable to install static systems over opaque facades, and keep the adaptive system for glazed facades only.\\

The design of an ASF naturally can greatly influence the results. Varying factors such as the choice of actuators and the complexity of the control system can change the emission factor. The largest variable however is the emission factor of the grid electricity mix. The building operational savings in heating, cooling, and lighting will have a CO${_2}$ saving based on the grid electricity mix. 
%Placing the ASF in a country with a low CO${_2}$ grid mix, such as Switzerland, will create an emission factor of 63.1 gCO${_2}$/kWh for the ASF. In Germany on the other hand, the emission factor will be -792.9 gCO${_2}$/kWh for the ASF.\\

Future research will validate the assumptions to building energy consumption through experimentation. This will be conducted on the ETH House of Natural Resources living lab where an example of an ASF has already been constructed \cite{nagy2015frontiers}. Further numerical simulations of the ASF on different building typologies, building systems and climates will enable us to specifically target the best application scenario. An application of the ASF in Spain for example could yield even larger energy savings compared to a temperate climate like Switzerland.\\

To conclude, we demonstrated that BIPV systems and adaptive shading elements complement each other successfully. We see an improvement in environmental performance of the PV technology, and create new architectural possibilities for the aesthetic integration of PV panels over glazed building surfaces, thus expanding BIPV potential. 

%The END




% The use of soft robotic actuators over servo motors increases the environnmental performance by \textcolor{red}{60 kgCO2} per square meter. The control electronics, which represents 11\% of the total embodied emissions, can increase by \textcolor{red}{35\%} if we increase the resolution of the facade so each panel can move independently. Also the supporting structure, representing \textcolor{red}{20\%} of total emissions can be better optimised to use less steel, or an alternative material. \\