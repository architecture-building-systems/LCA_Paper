% !TEX root = main.tex

The environmental performance of the ASF has been shown to be favorably competitive with a classic BIPV installation. Even though the embodied environmental performance is \textcolor{red}{six times} worse than a static CIGS installation, it is offset three fold through adaptive shading. This combination of adaptive shading with photovoltaic generation brings new advantages to solar as it effectively has a negative emission factor of \textcolor{red}{-470.1 g CO2/kWh}. These advantages however, will not be present if the ASF is installed over an opaque building surface. It is therefore better to install static systems over opaque facades, and keep the adaptive system for glazed facades only. \\

The design of an ASF naturally can greatly influence the results. Varying factors such as the choice of actuators and the complexity of the control system can change our emission factor by \textcolor{red}{XX\%}.\\
The largest variable however is the emission factor of the grid electricity mix. The building operational savings in heating, cooling, and lighting will have a CO2 saving based on the grid electricity mix. In a country with low CO2 grid mix, such as Switzerland, will have an emission factor of \textcolor{red}{140gCO2/kWh}. In Germany on the other hand, will have an emisison factor of \textcolor{red}{-700gCO2/kWh}.\\

We see that the combination of BIPV systems and adaptive shading, compliment each other nicely. We see an improvement in environmental performance of the PV technology, and open up new architectural possibilities for the aesthetic integration of PV panels over glazed building surfaces. 

The next steps of this research will validate the numerical simulations experimentaly. This will be conducted on the HoNR living lab where an example of an ASF has already been constructed \cite{nagy2015frontiers}. Further numerical simulations of the ASF on different building typologies, building systems and climates will yield interesting results. An application of the ASF in Spain for example could yield even larger energy savings compared to a temperate climate like Switzerland.

%The END


% The use of soft robotic actuators over servo motors increases the environnmental performance by \textcolor{red}{60 kgCO2} per square meter. The control electronics, which represents 11\% of the total embodied emissions, can increase by \textcolor{red}{35\%} if we increase the resolution of the facade so each panel can move independently. Also the supporting structure, representing \textcolor{red}{20\%} of total emissions can be better optimised to use less steel, or an alternative material. \\