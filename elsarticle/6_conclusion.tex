% !TEX root = main.tex

The environmental performance of the ASF has been shown to be favorably competitive with a classic BIPV installation. Even though the embodied environmental performance is \textcolor{red}{six times} worse than a static CIGS installation, it is offset three fold through adaptive shading. This combination of adaptive shading with photovoltaic generation brings new advantages to solar as it effectively has a negative emission factor of \textcolor{red}{444? g CO2/kWh}. However these advantages will not be present if the ASF is installed over an opaque building facade. It is therefore better to install static systems over opaque facades and keep the adaptive system for glazed facades. \\

The design of an ASF naturally can greatly influence the results. The use of soft robotic actuators over servo motors increases the environnmental performance by xxx per square meter. Also, control electronics represents XX\% of the total embodied emissions. An increase in complexity of the control system can quickly degrade the environmental performance. \\
The emission factor of the grid electricity mix also varies greatly between countries and has a large influence on the savings through adaptive shading. The reductions in heating, cooling and lighting of an office room will have CO2 savings based on the electricity mix. In a country with low CO2 grid mix will only see YY reductions. In Germany on the otherhand, there will be savings of YY.




- xxx\% of Embodied emissions of the photovoltaic BOS can be offset through smart shading\\
- This multi functionality brings about new advantages/disadvantages for solar as it has a reduced/increased the emissions per kWh by xxx\% \\
- Higher embodied CO2 compared to a classic photovltaic retrofit. However reduction can be made through x y and z\\
- Results are highly sensitive to x y and z\\
- Have a technology where you can put PV where you normally can't put PV