% !TEX root = main.tex

The analysis looks at the embodied carbon in production, operation, and disposal of the Adaptive Solar Facade (ASF).

\subsection{Embodied Energy Inventory and Assumptions}

The ASF is composed of six sub-product systems described in Figure \ref{fig:subsystem}. This consists of CIGS PV panels mounted on an actuator, supported by a cantilever that offsets it from a cable net supporting structure. An exploded view of these components can be seen in Figure \ref{fig:explodedView}.

\begin{figure}[H]
\begin{center}
\includegraphics[width=8cm, trim= 0cm 0cm 0cm 0cm,clip]{ASFSubsystems.pdf}
\caption{Breakdown of the ASF into six sub-product systems (Note change Steel frame to Suporting Structure, and Assembling ASF to Assembly. Also redraw this chart so it matches the subsubsections below)}
\label{fig:subsystem}
\end{center}
\end{figure}

\begin{figure}[H]
\begin{center}
\includegraphics[width=8cm, trim= 0cm 0cm 0cm 0cm,clip]{explodedASFV2.png}
\caption{Exploded view of an ASF module mounted on a cable net supporting structure}
\label{fig:explodedView}
\end{center}
\end{figure}

\subsubsection*{PV Panel}
The PV panels that can be used are restricted by their weight. Therefore, any technology that requires glass encapsulation or a heavy substructure cannot be used. This limits us to CIGS and amorphous silicon panels [Check this fact again]\\

CIGS PV panels was selected as the thin film panel of choice due to its high efficiency, low cost, and ability to be deposited on a polymer or aluminium substrate \cite{chirilua2011highly}. A less efficient thin film amorphous silicon panel could also be used and will also be discussed in this analysis.\\

\begin{table}[H]
\centering
\begin{tabular}{lll}
Panel Type		   & Embodied Carbon        & Efficiency     \\
\hline
CIGS 				 & XXg/m2 & YY\% \\
a-Si				 & YY g/m2        & YY\%    \\
\end{tabular}
\caption{Possible PV technologies for an ASF [Ref required]}
\label{tab:PV}
\end{table}

\begin{table}[H]
\centering
\begin{tabular}{ll}
\hline
CIGS         & xx g/m2 \\
Junction Box &         \\
Power Cables &         \\
\hline
\end{tabular}
\caption{Inventory of main input flows to the PV manufacturing process [ref required]}
\label{tab:PVinv}
\end{table}


\subsubsection*{Actuator}
Traditionally photovoltaic actuation is done through the use of servo motors. Servo motors however become a limiting factor for adaptive facades due to their high upfront costs, and instability in heavy winds. Soft robotic actuators on the other hand are cheaper and more resilient to harsh environmental conditions\cite{Svetozarevic2014a}. For the purpose of this analysis we will analyse both servo motors and soft robotic actuators. \\

\begin{table}[H]
\centering
\begin{tabular}{ll}
\hline
Compressor & xxg/unit  \\
Tubes      & xxgCO2/m  \\
Silicone   & xxgCO2/yy \\
\hline
\end{tabular}
\caption{Inventory of main input flows to the Actuator manufacturing process [ref required]}
\label{tab:ActuatorInv}
\end{table}

\subsubsection*{Cantilever}
The cantilever is a steel connection point between the PV panel and the supporting structure.\\

\begin{table}[H]
\centering
\begin{tabular}{ll}
\hline
Steel & xxgCO2/yy \\
xxxx  & xxgCO2/yy \\
yyyyy & xxgCO2/yy \\
\hline
\end{tabular}
\caption{Inventory of main input flows to the Cantilever manufacturing process [ref required]}
\label{tab:CantileverInv}
\end{table}

\subsubsection*{Supporting Structure}
The supporting structure is the connection point between the array of photovoltaic modules and the building itself. Many different designs are possible, however we will base our analysis of an adaptive solar facade that has already been constructed \cite{nagy2015frontiers}. This design consists of a steel cable-net that spans a steel supporting frame. The steel frame is then attached to the building itself.\\

\begin{table}[H]
\centering
\begin{tabular}{ll}
\hline
Steel & xxgCO2/yy \\
xxxx  & xxgCO2/yy \\
yyyyy & xxgCO2/yy \\
\hline
\end{tabular}
\caption{Inventory of main input flows to the manufacturing process of the Supporting Structure[ref required]}
\label{tab:StructureInv}
\end{table}

\subsubsection*{Controls and Electronic System}
The control system is required for the actuation of panels and the regulation of photovoltaic electricity production.\\

\begin{table}[H]
\centering
\begin{tabular}{ll}
\hline
Steel & xxgCO2/yy \\
xxxx  & xxgCO2/yy \\
yyyyy & xxgCO2/yy \\
\hline
\end{tabular}
\caption{Inventory of main input flows to the manufacturing process of the Control System[ref required]}
\label{tab:ControlInv}
\end{table}

\subsubsection*{Installation}

The installation of the ASF to the building requires a hydraulic hoist which needs to be in operation for eight hours based off previous construction experience \cite{jayathissa2015abs}. \\

\begin{table}[H]
\centering
\begin{tabular}{ll}
\hline
Steel & xxgCO2/yy \\
xxxx  & xxgCO2/yy \\
yyyyy & xxgCO2/yy \\
\hline
\end{tabular}
\caption{Inventory of main input flows to the Assembly Process[ref required]}
\label{tab:AssemblyInv}
\end{table}


\subsection{Operational Emissions and Assumptions}

The potential savings are based off previously completed numerical simulations \cite{jayathissa2015abs}. The simulation was conducted on a south facing office room. The room xx meters in length, xx meters wide and xx meters high was modeled using Rhinoceros 3D CAD Package \cite{Rhino}, shown in Figure XX. Grasshopper \cite{grasshopper} was used to model the dynamic aspects of the ASF which consists of an array of 400mm CIGS solar panels. The geometrical input is imported to Energy Plus \cite{energyplus} though the DIVA \cite{DIVA} interface. A single zone thermal analysis was conducted for each possible geometrical configuration of the ASF for each hour of the year. The results were then post processed in Python \cite{python} with the NumPy \cite{numpy}, and pandas \cite{pandas} plug-ins. \\

Based on the assumption of XX full openings and closings per day, we approximate the energy requirement to actuate the ASF to be YY kWh in its lifetime.\\

\begin{table}[H]
\centering
\begin{tabular}{ll}

\textbf{Building Settings}    &                                                \\
\hline
Office Envelope               & Roof: Adiabativ                                \\
                              & Floor: Adiabatic                               \\
                              & Walls: Adiabatic                               \\
                              & Window: Double Glazed LoE (e=0.2) 3mm/13mm air \\
                              & Floor Area: 21.7m2                             \\
\hline                           
Thermal Set Points            & Heating: 22 degrees Celcius                    \\
                              & Cooling: 26 degrees Celcius                    \\
\hline
Lighting Control              & Lighting set point: 11.8W/m2                   \\
                              & Lighting Control: 300 Lux Threshhold           \\
\hline
Occupancy                     & Office: Weekdays from 8:00-18:00               \\
                              & People set point: 0.1 persons/m2               \\
                              & Infiltration: 0.5 per hour                     \\
                              &                                                \\

\textbf{Location Assumptions} &                                                \\
\hline
Weather File                  & Geneva, Switzerland                            \\
Electricity Mix               & UCTE                                           \\
                              &                                                \\

\textbf{Maintenance}          &                                                \\
\hline
xxxx                          & xxxx                                           \\
                              &                                                \\

\textbf{ASF Settings}         &                                                \\
\hline
Full open and closes per day  &   yy                                           \\
\hline
\end{tabular}
\caption{Summary of main assumptions for the calculation of operational emissions [ref required]}
\label{tab:AssumptionsOpp}
\end{table}


% Maybe have a reference case here, see previous commits 

\subsection{Evaluation Method (Under heavy Construction)}



\begin{itemize}
\item The analysis is performed according to ISO 14040, ISO 14044 and ISO 15804. % 15804 to be discussed
	\item The impact category, which will be evaluated, is the global warming potential (GWP). This is described as the emissions of ${\mathrm{CO_2-eq}}$ in kilograms divided by the functional unit.
	\item The functional unit used is twofold and based on the function of the adaptive building envelope. For the comparison with other shading systems facade area in ${\mathrm{m^2}}$ is used, while comparison with other photovoltaic systems is done using electricity produced in ${\mathrm{kWh}}$. According to the guidelines of the International Energy Agency (IEA), the calculation of kWh produced needs to be based for consistency on conversion efficiency ${\eta}$, performance ratio PR, irradiation I, lifetime LT and area A of the module. Equation \ref{eq:solar} gives the exact formulation:
	% Variables in italics?
	% LT - service life?
\begin{equation}
G=\frac{{\mathrm{GWP}}}{{\mathrm{I \cdot \eta  \cdot PR \cdot LT \cdot A}}}
% what is G
\label{eq:solar}
\end{equation}

	\item The LCI inventory was obtained through...

	\item The scope of the LCA comprises the embodied, operational and disposal global warming impact of the respective system. Figure \ref{fig:BOS} illustrates the system boundaries of the process flows. The supporting structures are also included in the system boundaries. The reason for this is that technologies within the building envelope also change the design of the supporting structures. The supporting structure of solar panels is referred to as balance of systems (BOS).
% We will need to describe a little more what is included and what is not, i.e.
%   not only supporting structure

\begin{figure}[H]
\begin{center}
\includegraphics[width=5cm, trim= 0cm 0cm 0cm 0cm,clip]{BOS}
\caption{Thin-film incl. BOS system boundaries}
\label{fig:BOS}
\end{center}
\end{figure}

	\item The cut-off approach is used for recycling and landfill. This means that recycling does not generate any credit for the product and resulting benefits are not taken into account. Furthermore the use of recycled products do not bear the burden of processes higher up the chain.
	% we may need to discuss system expansion
	% PV electricity production not included?
	\item The recipe midpoint (H) allocation method allows for an accurate evaluation of the GWP based on human impact factors.
	% I don't get it - sorry
	% I am pretty sure ReciPe basically uses the IPCC method
\end{itemize}



% what LCI DB (ecoinvent) is used? refer to Annex?
% How was LCI data collected?
% see above. Explain what is included and excluded
