% !TEX root = main.tex

- The results of the analysis can be summarized in Figure \ref{fig:waterfall}...

\begin{figure}[H]
\begin{center}
\includegraphics[width=10cm, trim= 0cm 0cm 0cm 0cm,clip]{waterfall}
\caption{Breakdown of GWP of the ASF into embodied, operational and disposal emissions}
\label{fig:waterfall}
\end{center}
\end{figure}
% sorry - not self-explanatory for me...

- A breakdown of the embodied carbon emissions can be found in Figure  \ref{fig:embodied}... (brief discussion and design consideration)...

\begin{figure}[H]
\begin{center}
\includegraphics[width=10cm, trim= 0cm 0cm 0cm 0cm,clip]{embodied}
\caption{Breakdown of the embodied carbon emissions, it can be seen that xxxx has the greatest GWP contribution}
\label{fig:embodied}
\end{center}
\end{figure}



- As input parameters of production processes are stochastic, a Monte Carlo simulation is used to include this stochastic behavior in the results, as shown in Figure \ref{fig:monte}...

\begin{figure}[H]
\begin{center}
\includegraphics[width=10cm, trim= 0cm 0cm 0cm 0cm,clip]{monte}
\caption{Monte carlo simulation based on input uncertainties}
\label{fig:monte}
\end{center}
\end{figure}

- Sourcing location greatly influences the embodied GWP. For photovoltaic panels, the majority of embodied emissions result from the use of electricity during production. The GWP per kwh of the Chinese electricity mix is 1145.8 ${\mathrm{gCO_2eq/kWh}}$, while in Switzerland this is only 119.6 ${\mathrm{gCO_2eq/kWh}}$.
% cool - did we already discuss what other sensitivities to use?

\begin{figure}[H]
\begin{center}
\includegraphics[width=10cm, trim= 0cm 0cm 0cm 0cm,clip]{sensitivity}
\caption{Sensitivity analysis based on sourcing location}
\label{fig:sensitivity}
\end{center}
\end{figure}
