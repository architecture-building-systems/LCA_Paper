\documentclass[a4paper,12pt]{letter}

% \usepackage{fancyhdr}
% \usepackage{pagecounting}
% \usepackage[dvips]{color}
% \definecolor{gray}{rgb}{0.4,0.4,0.4}
% \usepackage{blindtext} % for dummy text
%
% \usepackage{hyperref}
% \usepackage{lastpage}
\usepackage[margin=3cm]{geometry}   % to change margins
\pagestyle{empty} % no page numbering
\usepackage{blindtext} % lorem-ipsum text
\newcommand{\amper}{{\selectfont\itshape\&}}


\address{ETH Zurich \\ Zurich 8093 \\ Switzerland \\ jayathissa@arch.ethz.ch}
\begin{document}

\begin{letter}{Editorial Office \\ Journal of Solar Energy Materials and Solar Cells\\ Special Issue: Life cycle, environmental, ecology and impact analysis of solar technology}
	\opening{Dear Dr. Espinosa and Prof. Dr.(?) Krebs,}
	We are pleased to submit our manuscript entitled \emph{Life Cycle Assessment of Dynamic Building Integrated Photovoltaics} to this special issue of the Journal of Solar Energy Materials and Solar Cells. 
	Our manuscript assesses the field of dynamic building integrated photovoltaic (BIPV) systems, which is gaining interest due to the increasing efficiencies of light weight thin film PV technologies. 

	%The increasing efficiency of light weight, thin film photovoltaic technologies, has enabled a growth in interest of dynamic BIPV systems. This is because dynamic BIPV systems can combine the benefits of adaptive building shading, with solar tracking. 

	%This study investigates the environmental performance of dynamic BIPV systems. In particular to assess whether the additional material and actuation costs are offset by the the savings to the buildings energy demand. 

	We find that the environmental performance of a dynamic BIPV system is beneficial if it is built over glazed building surfaces, because the building profits from both adaptive shading, and PV production through solar tracking. For opaque facades, we see siple static BIPV solutions more favorable than dynamic systems. Further, we present considerations for the design of dynamic BIPV systems.
	
	We believe our manuscript is of great interest to the researchers, engineers and architects reading your journal, as our findings can be readily applied in practice.
	%Our findings can be readily applied in practice, they are likely to be of great interest to the vision of researchers, engineers and architects who read your journal. 

	This manuscript describes original work and is not under consideration by any other journal. All authors approved the manuscript and this submission

	Thank you for receiving our manuscript and considering it for review. We appreciate your time and look forward to your response

	\signature{Prageeth Jayathissa}
	\closing{Yours sincerely,}
\end{letter}

	



\end{document}